\documentclass{article}

\usepackage{polski}
\usepackage[utf8]{inputenc}
\usepackage{geometry}
	\geometry{a4paper}
\usepackage{array}
%\usepackage{changepage}
\usepackage{enumitem}
\usepackage{longtable}

\begin{document}
\begin{flushleft}
Zaplanowane testy sprawdzające funkcjonalność zawartą w widoku okna aplikacji. \\
(W-wariant, R-rekurencja, NULL-brak, ,,w każdym widoku''- w widoku dnia, tygodnia i mies.)
\setlength\LTleft{-1in}

	\begin{longtable}[H]{| m{3.5cm} | m{3.5cm} | m{3.5cm} | m{3.5cm} | m{3.5cm} |} \hline
	Nazwa & Warunki wstępne & Kroki wykonania & Oczekiwany rezultat & Otrzymany rezultat \\ \hline

	Zmiany strefy czasowej & Użytkownik znajduje się aktualnie w widoku dnia. & \begin{enumerate}[leftmargin =*, topsep=0pt] \item Otworzenie okna zmiany strefy czasowej. \item 
	Wybór dowolnej ze stref czasowych. \end{enumerate} & Zmiana wyświetlanej strefy czasowej. Przesunięcie utworzonych wydarzeń zgodnie ze zmianą czasu. Jeśli zajdzie taka potrzeba, 
	przesunięcie wydarzeń na poprzedni/następny dzień. & - \\ \hline

	 Rozwinięcie rozpisu godzin & Użytkownik znajduje się aktualnie w widoku dnia. & \begin{enumerate}[leftmargin =*, topsep=0pt] \item Kliknięcie w strzałkę rozwijającą widok dnia. 
	\end{enumerate} & Rozwinięcie widoku dnia. & - \\ \hline

	Przejście do poprzedniego dnia W1 & Użytkownik znajduje się aktualnie w widoku dnia. Aktualny dzień nie jest pierwszym dniem zbioru dni. & \begin{enumerate}[leftmargin =*, topsep=0pt] \item 
	Kliknięcie w strzałkę przejścia do dnia poprzedniego \end{enumerate} & Przejście do widoku dnia poprzedniego. Zmiana daty i listy aktywności. & - \\ \hline

	Przejście do poprzedniego dnia W2 & Użytkownik znajduje się aktualnie w widoku dnia. Aktualny dzień jest pierwszym dniem zbioru dni. & \begin{enumerate}[leftmargin =*, topsep=0pt] \item 
	Kliknięcie w strzałkę przejścia do dnia poprzedniego. \end{enumerate} & Pozostanie w obecnym widoku dnia. & - \\ \hline
	
	Przejście do następnego dnia W1 & Użytkownik znajduje się aktualnie w widoku dnia. Aktualny dzień nie jest ostatnim dniem zbioru dni. & \begin{enumerate}[leftmargin =*, topsep=0pt] \item 
	Kliknięcie w strzałkę przejścia do dnia następnego. \end{enumerate} & Przejście do widoku dnia następnego. Zmiana daty i aktywności. & - \\ \hline

	Przejście do następnego dnia W2 & Użytkownik znajduje się aktualnie w widoku dnia. Aktualny dzień jest ostatnim dniem zbioru dni. & \begin{enumerate}[leftmargin =*, topsep=0pt] \item Kliknięcie 
	w strzałkę przejścia do dnia następnego. \end{enumerate} & Pozostanie w obecnym widoku dnia. & - \\ \hline

	Dodanie aktywności & Użytkownik znajduje się aktualnie w widoku dnia. &  \begin{enumerate}[leftmargin =*, topsep=0pt] \item Kliknięcie w klawisz dodawania aktywności. \item 
	Dodanie aktywności bez zaznaczonej opcji "recurring". \end{enumerate} & Zamknięcie się okna dodawania aktywności. Dodanie do planu (w każdym widoku) stworzonej aktywności i umieszczenie 
	jej przy odpowiedniej godzinie. & -  \\ \hline

	Dodanie aktywności R & Użytkownik znajduje się aktualnie w widoku dnia. &  \begin{enumerate}[leftmargin =*, topsep=0pt] \item Kliknięcie w klawisz dodawania aktywności. \item 
	Dodanie aktywności z zaznaczoną opcją "recurring". \end{enumerate} & Zamknięcie się okna dodawania aktywności. Dodanie do planu (w każdym widoku) stworzonej aktywności i umieszczenie jej 
	przy odpowiedniej godzinie, dodanie aktywności w każdym następnym dniu o tej samej nazwie (np. w każdy pon). & -  \\ \hline

	Edycja aktywności & Użytkownik znajduje się aktualnie w widoku dnia. Istnieje co najmniej jedna aktywność. & \begin{enumerate}[leftmargin =*, topsep=0pt] \item Kliknięcie w klawisz edycji 
	aktywności. \item Wybór aktywności do edycji. \item Dokonanie zamierzonej edycji bez zaznaczenia opcji "recurring". \end{enumerate} & Zamknięcie się okna edycji aktywności. Naniesienie 
	poprawek do informacji zawartych w edytowanej aktywności. & - \\ \hline

	Edycja aktywności R1 & Użytkownik znajduje się aktualnie w widoku dnia. Istnieje co najmniej jedna aktywność, która została stworzona z opcją "recurring". & \begin{enumerate}[leftmargin =*, 
	topsep=0pt] \item Kliknięcie w klawisz edycji aktywności. \item Wybór aktywności do edycji. \item Dokonanie zamierzonej edycji z zaznaczeną opcją "recurring". \end{enumerate} & Zamknięcie się 
	okna edycji aktywności. Naniesienie poprawek do informacji zawartych w edytowanej aktywności i każdej instancji tej aktywności stworzonej przez opcję "recurring".  & - \\ \hline

	Edycja aktywności R2 & Użytkownik znajduje się aktualnie w widoku dnia. Istnieje co najmniej jedna aktywność, która nie została stworzona z opcją "recurring". & \begin{enumerate}[leftmargin =*, 
	topsep=0pt] \item Kliknięcie w klawisz edycji aktywności. \item Wybór aktywności do edycji. \item Dokonanie zamierzonej edycji z zaznaczeną opcją "recurring". \end{enumerate} & Zamknięcie się 
	okna edycji aktywności. Naniesienie poprawek do informacji zawartych w edytowanej aktywności i utworzenie aktywności z naniesionymi poprawkami w każdym następnym dniu o tej samej nazwie 
	(np. w każdy pon).  & - \\ \hline

	Edycja aktywności NULL & Użytkownik znajduje się aktualnie w widoku dnia. W danym dniu nie ma zaplanowanej żadnej aktywności. & \begin{enumerate}[leftmargin =*, topsep=0pt] \item Kliknięcie 
	w klawisz edycji aktywności. \end{enumerate} & Wyświetla się pusta lista możliwych do edycji aktywności, nie pozwalając użytkownikowi na kontynuację czynności. & - \\ \hline

	Usuwanie aktywności & Użytkownik znajduje się aktualnie w widoku dnia. Istnieje co najmniej jedna aktywność. & \begin{enumerate}[leftmargin =*, topsep=0pt] \item Kliknięcie w klawisz usunięcia 
	aktywności. \item Wybór aktywności do usunięcia bez zaznaczenia opcji "recurring". \end{enumerate} & Usunięcie wybranej aktywności z aktualnego dnia w każdym z widoków. & - \\ \hline

	Usuwanie aktywności R & Użytkownik znajduje się aktualnie w widoku dnia. Istnieje co najmniej jedna aktywność. & \begin{enumerate}[leftmargin =*, topsep=0pt] \item Kliknięcie w klawisz 	
	usunięcia aktywności. \item Wybór aktywności do usunięcia z zaznaczeniem opcji "recurring". \end{enumerate} & Usunięcie wybranej aktywności z aktualnego dnia oraz każdej instancji tej 
	aktywności w każdym następnym dniu o tej samej nazwie (np. w każdy pon) (o ile takie istnieją) w każdym z widoków. & - \\ \hline

	Usuwanie aktywności NULL & Użytkownik znajduje się aktualnie w widoku dnia. W danym dniu nie ma zaplanowanej żadnej aktywności. &   \begin{enumerate}[leftmargin =*, topsep=0pt] \item 
	Kliknięcie w klawisz edycji aktywności. \end{enumerate} & Wyświetla się pusta lista możliwych do usunięcia aktywności, nie pozwalając użytkownikowi na kontynuację czynności. & - \\ \hline

	Powiadomienie o święcie W1 & Użytkownik znajduje się aktualnie w widoku dnia. Przeglądany dzień nie jest uznawany za jakiekolwiek święto. & - & Kwadrat odpowiadający za powiadomienie jest 
	"zgaszony". & - \\ \hline

	Powiadomienie o święcie W2 & Użytkownik znajduje się aktualnie w widoku dnia. Przeglądany dzień jest uznawany za jakiekolwiek święto. & - & Kwadrat odpowiadający za powiadomienie jest 
	"zapalony". & - \\ \hline
	
	Zmiana widoku- dzień & Użytkownik znajduje się aktualnie w widoku dnia. & \begin{enumerate}[leftmargin =*, topsep=0pt] \item Kliknięcie w przycisk zmiany trybu widoku. \item Wybór trybu 
	dnia. \end{enumerate} & - & - \\ \hline
	
	Zmiana widoku- tydzień & Użytkownik znajduje się aktualnie w widoku dnia. & \begin{enumerate}[leftmargin =*, topsep=0pt] \item Kliknięcie w przycisk zmiany trybu widoku. \item Wybór trybu 
	tygodnia. \end{enumerate} & Przejście do widoku aktualnego tygodnia. & - \\ \hline
	
	Zmiana widoku- miesiąc & Użytkownik znajduje się aktualnie w widoku dnia. & \begin{enumerate}[leftmargin =*, topsep=0pt] \item Kliknięcie w przycisk zmiany trybu widoku. \item Wybór trybu 
	miesiąca. \end{enumerate} & Przejście do widoku aktualnego miesiąca. & - \\ \hline
	\end{longtable}


\end{flushleft}
\end{document}